\documentclass[a4paper,10pt]{article}
\usepackage[english]{babel}
\usepackage[utf8]{inputenc}
\usepackage{pict2e}
\usepackage{colortbl}
\usepackage{graphicx}
\usepackage{amssymb}
\usepackage{amsmath}
\usepackage{latexsym}
\usepackage{amsthm}
\usepackage{times}
\usepackage{booktabs}
\usepackage{multirow}
\usepackage{slashed}
\usepackage{listings}
\usepackage{hyperref}
\usepackage[hdivide={2.5cm, *, 2.5cm}, vscale=0.9, bindingoffset=0cm]{geometry}
\linespread{1.1}
\pagestyle{empty}

\definecolor{myblue}{rgb}{0.20,0.40,0.65}
\input{../Physics/mycommands.tex}

\begin{document}

\begin{center}
\LARGE\color{myblue}
\textsc{Statement of Research Interest}
\end{center}

\begin{flushright}
\Large
\textsc{Antonella Succurro}
\end{flushright}

2010 was an incredibly lucky year to start a PhD in High Energy Particle Physics. The Large Hadron Collider had just started delivering 
proton-proton collisions to the experiments with an efficiency in performance that kept rising month after month. Three years of data-taking
with increased center of mass energies allowed the ATLAS collaboration to collect enough events to obtain ground-breaking physics results, the discovery
of a new boson being undoubtedly the most advertised one. Looking now at the foreseen upgrades for the LHC and its experiments we
can say that this was just the beginning. Too many questions remain unanswered, even after three years of such high-level performance
in all areas: hardware, machine operation, software development and, last but not least, physics analysis.

How can we understand what lies behind the high-energy frontier? New physics has turned out to be better hidden than anticipated.
Many efforts have already been spent in trying to find some hints of beyond-Standard Model physics but we are still blind.
Which basically means: keep on searching.

\subroottitle*{\color{myblue}New Physics Searches}

But \textit{what} exactly is ``new physics''? Some models proposed by theorists are better motivated than others, 
but in the end we do not really have a clue on where exactly will new physics manifest itself, nor what form will it take. This is why
I find very interesting those searches that are sensitive to many possible signals. There are particular features
of final states that are common to beyond-Standard Model scenarios, which means that by relaxing or enhancing certain
selections we can be sensitive to more than just one specific signal. As an example, a large variety of new particles
would leave a signature similar to $t\bar{t}\to W^+W^-b\bar{b}$. In the short term I would like to profit from the experience
gained with the $gg\to T\bar{T}\to Ht+X\to b\bar{b}l\nu b +X$ analysis~\cite{ATLAS-CONF-2013-018}, where a final state 
involving one lepton, missing transverse energy, many jets (six or more) and many
$b$-tagged jets (four or more) has been shown to be very sensitive not only to vector-like top-partners 
(predicted in Composite Higgs models and in Extra-dimensions theories) but also to other possible new physics signals such as
top quark compositeness or R-Parity Violating SUSY~\cite[p.17,22]{Yevgeny}.

In particular for the long term, I would like to focus on searches for single production of vector-like quarks~\cite{JuanAntonio}, again keeping 
in mind possible re-interpretations to have an analysis as model independent as possible. For heavy quarks masses of $\mathcal{O}(1~\tev)$
electroweak single production can be more relevant than strong pair production, plus sensitivity will increase since the production cross
roottitle raises by a higher factor with respect to pair production at the center-of-mass energies foreseen for LHC Run II.
In addition to what is my current physics interests, I am eager to join other physics projects to which I can contribute
with my skills and from which I can gain experience.


\subroottitle*{\color{myblue}Detector Upgrades}

During the long shutdown the ATLAS detector will be significantly upgraded in order to cope with the new challenges it will face
at the higher luminosities that will be delivered by the LHC. One of the main priorities will be to cope with an increased pile-up and maintain
the excellent performance in object reconstruction of Run~I. Tracking particles that will flow at higher rates through the detector
systems will be a key issue. I would like to join the effort in testing and characterizing the Micromegas that will be included in the
new small wheels to improve trigger and tracking performance in the End-Cap region. I find this technology particularly interesting
also because of its possible applications in other scientific fields like medical physics.

\subroottitle*{\color{myblue}Teaching and Outreach}

As a CERN guide I learned how important and fulfilling transfer of knowledge is. This is why during my
postdoc appointment I would gladly join outreach programs as well as
teaching/tutoring projects.



\myskip
\vfill

\line(1,0){250}
\scriptsize
\subsubroottitle*{\color{myblue}References}
\begingroup
\renewcommand{\roottitle}[2]{}%
\begin{thebibliography}{9}

\bibitem{ATLAS-CONF-2013-018}
  ATLAS collaboration,
  \emph{Search for heavy top-like quarks decaying to a Higgs boson and a top quark in the lepton plus jets final state in $pp$ collisions at $\sqrt{s}=8$ TeV with the ATLAS detector},
  \href{http://atlas.web.cern.ch/Atlas/GROUPS/PHYSICS/CONFNOTES/ATLAS-CONF-2013-018}{ATLAS-CONF-2013-018}

\bibitem{Yevgeny}
  Y. Kats,
  \emph{LHC searches examined via the RPV MSSM},
  \href{http://indico.cern.ch/contributionDisplay.py?sessionId=15&contribId=668&confId=218030}{EPS talk}

\bibitem{JuanAntonio}
  J.A. Aguilar-Saavedra, R. Benbrik, S. Heinemeyer, M. Perez-Victoria,
  \emph{A handbook of vector-like quarks: mixing and single production},
  \href{http://arxiv.org/abs/1306.0572}{arXiv:1306.0572}


\end{thebibliography}
\endgroup

\end{document}
