
\roottitle{Education}

\headedsection{\textsc{Universitat Autònoma de Barcelona (UAB)}}{Spain} {

\headedsubsection{\color{myblue} PhD in Particle Physics}{2014-02-28, excellent cum Laude}

\ifIsAcademic

{\bodytext{\small%\footnotesize
I worked on-site at the CERN laboratories in Geneva for most of the PhD period. The final dissertation titled 
\textit{Probing new physics at the LHC: searches for heavy top-like quarks with the ATLAS experiment} was written under the supervision of Prof. Aurelio Juste. Available at \url{http://www.tdx.cat/handle/10803/133340}. The analysis developed brought to the publication \url{http://dx.doi.org/10.1007/JHEP08(2015)105}. During my contribution period the most significant outcome from the ATLAS collaboration was the discovery of the Higgs boson in 2012 (\url{http://dx.doi.org/10.1126/science.1232005}).
}}

\fi

}


\headedsection{\textsc{Università degli Studi di Pavia \& Institute for Advanced Study IUSS Pavia}}{Italy} {

\headedsubsection{\color{myblue} Master (Laurea Specialistica) in Particle Physics}{2009-12-18, 110/110 cum Laude}
{}

\headedsubsection{\color{myblue} IUSS Diploma (Alumn of Collegio Ghislieri)}{2010-05-18, Excellent}

\ifIsAcademic

{\bodytext{\small 
The Master thesis titled \textit{Searches for SUSY signals at the Large Hadron Collider - Light Stop Analysis} was written under the supervision of Dr. Giacomo Polesello and most of the work performed on-site at the CERN laboratories. 
Since 2004 I was Alumn of Collegio Ghislieri (\url{http://www.ghislieri.it/}) and of the Institute for Advanced Study IUSS Pavia (\url{http://www.iusspavia.it/eng/index.php}). The IUSS Diploma dissertation titled 
\textit{Superparticles mass measurement methods at the LHC} was written
under the supervision of Dr. Giacomo Polesello and Prof. Giorgio Goggi.
}}

\fi

}


%% \headedsection{\textsc{Università degli Studi di Pavia}}{Italy} {
%% \headedsubsection{\color{myblue} 3-year Bachelor (Laurea Triennale) in Physics}{2007-12-14, 110/110}

%% \ifIsAcademic

%% {\bodytext{\small %Decadimenti Deboli dei Quark: la Matrice CKM (
%% The Bachelor thesis titled \textit{Weak Decays of Quarks: the CKM Matrix} was written under the supervision of Prof. Claudio Conta.
%% }}

%% \fi

%% }

%% \headedsection{\textsc{Liceo Scientifico ``T. Taramelli'', Pavia}}{Italy} {
%% \headedsubsection{\color{myblue} High School Diploma (Maturit\`{a} Scientifica)}{2004-07-06, 100/100}

%% }


%%%%%%%%%%%%%%%




