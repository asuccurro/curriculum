
\roottitle{Research Activity}


\headedsection{\textsc{Systems Biology}}{May 2014 - present} {


\headedsubsection{\href{http://www.accliphot.eu/}{\color{myblue} AccliPhot}}{Heinrich Heine Universität Düsseldorf (Germany), }{\footnotesize Sep 2014 - present} 

{\bodytext{\small
How does the photosynthetic metabolic network in microalgae and plants adapt upon changes in external light conditions during short-time intervals?
My main research project within the Environmental Acclimation of Photosynthesis (AccliPhot) Marie Curie ITN
focuses on modeling the signaling pathways in plants and microalgae, in particular the key role of kinases and phosphatases enzymes. 
Furthermore, as a continuation of my previous project, I am investigating how algae and bacteria exchange nutrients and cofactors in 
mutualistic consortia, extending the scope of dynamic FBA to include the effect of metabolite concentrations on reaction fluxes.
}}


\headedsubsection{\color{myblue} Key achievements (expected)}{\color{myblue} Bioinformatics, Biochemistry Laboratory, R, dynamic FBA, Kinetic Modelling, Teaching}

\headedsubsection{\href{http://timing-metabolism.eu/}{\color{myblue} TiMet}}{University of Aberdeen (UK), }{\footnotesize May - Sep 2014} 


{\bodytext{\small
The european TiMet project aims at achieving a better understanding of how plants regulate their metabolism in response to the alternation of day and night.
During a very short term appointment I have been involved in the extention of a dynamic flux balance analysis software simulating microbial colony growth to plants.
}}


\headedsubsection{\color{myblue} Key achievements}{\color{myblue} Biology, Flux Balance Analysis, Matlab, Biochemistry, Linear Programming}



}



\headedsection{\textsc{High energy particle physics}}{February 2010 - February 2014} {

\headedsubsection{\href{http://atlas.ch/}{\color{myblue} ATLAS experiment}}{{CERN (Switzerland) \& IFAE (Spain)}} 

{\bodytext{\small
Is our Standard Model (SM) of the fundamental particle interactions complete? Apparently, the answer is ``no''. Many theories have been proposed to explain what is currently not understood, like the nature of Dark Matter, or the reason why the Higgs boson is so light. With data from the Large Hadron Collider (LHC) at the CERN laboratory of Geneva, the ATLAS experiment can probe new physics. I worked on various analyses aimed at the discovery (or exclusion) of a signal from a new quark similar to the top quark but with a larger mass. The results were the first to propose a model-independent way for comprehensive searches of vector-like quarks.
 I was involved also in other projects, in particular I have contributed to various performance studies of the ATLAS Tile Hadronic Calorimeter and for several months I have been responsible for the optimization of data-driven techniques to estimate the contribution of multi-jet events in a particular search channel used by many analyses of the ATLAS collaboration.}}

\color{myblue}
\headedsubsection{Key achievements}{Monte Carlo techniques, Statistical Analysis, 
ROOT, C++, Python, Bash scripting, RCS, 
Team working, Scientific writing, Written and Oral presentation skills, 
TWiki documentation}

}

